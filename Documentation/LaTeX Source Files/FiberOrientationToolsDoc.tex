\documentclass[11pt]{article}
\usepackage{fullpage}
\usepackage{amsmath, amssymb}

\title{An Overview of \\
      \textbf{Fiber Orientation Tools} \\
      { \normalsize \texttt{http://github.com/charlestucker3/Fiber-Orientation-Tools}} }

\author{Charles L.~Tucker III \\
       Department of Mechanical Science and Engineering \\
        University of Illinois at Urbana-Champaign \\
        1206 W.~Green St. \\
        Urbana, IL 61801
        }
% Definitions and declarations for "Fundamentals of Fiber Orientation" by C. L. Tucker

\DeclareMathOperator{\tr}{tr}   % Math operator for the trace of tensor (must be in the preamble)

% Math shortcuts j
\newcommand{\ud}{\mathop{}\!\mathrm{d}}  % Roman d as the differential operator,  no space after
\newcommand{\uD}{\mathop{}\!\mathrm{D}}  % Roman D as the differential operator for D/Dt
\newcommand{\A}{\ensuremath{\mathbf{A}}}          % Second-order orientation tensor, direct notation
\newcommand{\Afour}{\ensuremath{\mathbb{A}}}  % Fourth-order orientation tensor
\newcommand{\B}{\ensuremath{\mathbf{B}}}           % Piola tensor, direct notation
\newcommand{\Bfour}{\ensuremath{\mathbb{B}}}   % Fourth-order strain concentration tensor 
\newcommand{\C}{\ensuremath{\mathbf{C}}}          % Fiber-fiber interaction tensor in ARD models
\newcommand{\Cfour}{\ensuremath{\mathbb{C}}}  % Fourth-order stiffness tensor
\newcommand{\D}{\ensuremath{\mathbf{D}}}          % Rate-of-deformation tensor, direct notation
\newcommand{\Efour}{\ensuremath{\mathbb{E}}}   % Fourth-order Eshelby tensor
\newcommand{\F}{\ensuremath{\mathbf{F}}}            % Deformation gradient tensor, direct notation
\newcommand{\Ffour}{\ensuremath{\mathbb{F}}}    % Arbitrary fourth-order tensor (Appendix A) 
\newcommand{\Gfour}{\ensuremath{\mathbb{G}}}   % Arbitrary fourth-order tensor (Appendix A)  
\newcommand{\Hfour}{\ensuremath{\mathbb{H}}}   % Fourth-order strain concentration, Eshelby dilute model 
\newcommand{\Ifour}{\ensuremath{\mathbb{I}}}   % Fourth-order identity tensor
\newcommand{\jv}{\ensuremath{\mathbf{j}}}           % j vector (bold) for particle flux
\newcommand{\Lfour}{\ensuremath{\mathbb{L}}}   % Fourth-order L tensor (RSC model)
\newcommand{\Lub}{\ensuremath{L_\mathrm{ub}}} % Unbreakable fiber length L_ub
\newcommand{\Mfour}{\ensuremath{\mathbb{M}}}  % Fourth-order M tensor (RSC model)
\newcommand{\St}{\ensuremath{\mathbf{S}}}          % Arbitrary second-order tensor (Appendix A) 
\newcommand{\T}{\ensuremath{\mathbf{T}}}          % Arbitrary second-order tensor (Appendix A) 
\newcommand{\Tfour}{\ensuremath{\mathbb{T}}}  % Arbitrary fourth-order tensor, groupFlow chapter
\newcommand{\W}{\ensuremath{\mathbf{W}}}        % Vorticity, direct notation
\newcommand{\ev}{\ensuremath{\mathbf{e}}}         % e vector, bold
\newcommand{\fv}{\ensuremath{\mathbf{f}}}         % f vector, bold
\newcommand{\pv}{\ensuremath{\mathbf{p}}}         % p vector, bold
\newcommand{\mv}{\ensuremath{\mathbf{m}}}         % m vector, bold
\newcommand{\nv}{\ensuremath{\mathbf{n}}}         % n vector, bold
\newcommand{\qv}{\ensuremath{\mathbf{q}}}         % q vector, bold
\newcommand{\Rfour}{\ensuremath{\mathbb{R}}}  % 6x6 matrix used in fourth-order contracted notation 
\newcommand{\Sfour}{\ensuremath{\mathbb{S}}}   % Fourth-order compliance tensor 
\newcommand{\tv}{\ensuremath{\mathbf{t}}}          % t vector, bold
\newcommand{\uv}{\ensuremath{\mathbf{u}}}          % u vector, bold
\newcommand{\xv}{\ensuremath{\mathbf{x}}}          % x vector, bold
\newcommand{\vf}{\ensuremath{\phi_{f}}}                 % v_f (fiber volume fraction) with some kerning of the f
\newcommand{\vm}{\ensuremath{\phi_m}}                 % v_m (matrix volume fraction) 
\newcommand{\vv}{\ensuremath{\mathbf{v}}}          % v vector, bold
\newcommand{\dv}{\ensuremath{\skew{4} \hat{\boldsymbol{\delta}}}}  % delta-hat unit vector, bold
\newcommand{\jj}{\ensuremath{j}}                          % use in subscripts for better kerning of j
\newcommand{\ff}{\ensuremath{f}}                          % use in subscripts for better kerning of f
\newcommand{\epsilondotP}{\ensuremath \dot{\varepsilon}_{\! \scriptscriptstyle P}}  % epsilon-dot sub-P
\newcommand{\epsilondotB}{\ensuremath \dot{\varepsilon}_{\! \scriptscriptstyle B}}  % epsilon-dot sub-B
\newcommand{\epsilondotU}{\ensuremath \dot{\varepsilon}_{\! \scriptscriptstyle U}}  % epsilon-dot sub-U
\newcommand{\balpha}{\ensuremath{\boldsymbol{\alpha}}}                  % bold alpha, thermal expansion tensor
\newcommand{\bbeta}{\ensuremath{\boldsymbol{\beta}}}                      % bold beta, thermal stress tensor
\newcommand{\bepsilon}{\ensuremath{\boldsymbol{\varepsilon}}}       % bold epsilon, small strain tensor
\newcommand{\bsigma}{\ensuremath{\boldsymbol{\sigma}}}                  % sigma (total stress tensor), bold
\newcommand{\bSigma}{\ensuremath{\boldsymbol{\Sigma}}}                 % cap Sigma, bold
\newcommand{\btau}{\ensuremath{\boldsymbol{\tau}}}                           % tau (extra stress tensor), bold
\newcommand{\etaI}{\ensuremath \eta_{\scriptscriptstyle I}}                                         % eta sub-I
\newcommand{\elli}{\ensuremath \ell_{i}}                                                                       % ell sub-i
\newcommand{\omegav}{\ensuremath{\boldsymbol{\omega}}}              % bold omega, for angular velocity
\newcommand{\matlab}{\textsc{Matlab}}                   % "Matlab" in small caps
\newcommand{\nablasurf}{\ensuremath{\nabla_{s}}}    % surface gradient operator, w/ _s kerned


  % Macro definitions
\date{Version 1.1 \\June 20, 2022}  % *********** Change this when updating the documentation *******************

\begin{document}
\maketitle

This document summarizes \textbf{Fiber Orientation Tools}, a set of \matlab\ functions for modeling flow-induced fiber orientation in discontinuous fiber composites, and for predicting the resulting mechanical properties.  

The tools accompany the book \textit{Fundamentals of Fiber Orientation: Description, Measurement and Prediction} by C. L. Tucker III (Hanser, Munich, 2022), and references to sections, figures, and equations indicate items in the book.  

Also included are functions from the article ``Planar Fiber Orientation: Jeffery, Non-Orthotropic Closures and Reconstructing Distribution Functions,'' submitted to the \textit{Journal of Non-Newtonian Fluid Mechanics}, June 2022.

 \matlab\ live scripts that demonstrate the use of various tools are described first, followed by a list of the functions in the toolkit, organized by category.  To see the details of any function, type \texttt{help} followed by the function name in the \matlab\ command window.



\section{Live Scripts with Example Calculations}

The live scripts are arranged topically, following the chapters of the book.  Scripts related the planar orientation journal article are listed last.

\subsection*{Chapter 2.  Describing Fiber Orientation and Length}

\begin{description}
    
    \item[OrientationDistributionFunctions.mlx]{creates 3-D orientation distribution functions $\psi(\pv)$ using the Jeffery distribution function, as in Fig.~2.5.  Uses \texttt{A2F} and \texttt{drawPsi}.}
    
    \item[OrientationTensorExamples.mlx]{calculates orientation tensors for various combinations of $\pv$ vectors.  Uses \texttt{p2A} and follows the examples in Sections~2.3.1 and 2.3.5.}

    \item[EigenvaluesEigenvectors.mlx]{finds eigenvalues and eigenvectors of a second-order orientation tensor, and compares the standard \matlab\ function \texttt{eig} with the \texttt{eigsort} function from this toolkit.}
    
    \item[ReconstructDiscreteDistributionFcn.mlx]{shows how to find a set of orientation vectors $\pv^i$ and weighting factors $f_i$ to form a discrete approximation of an orientation distribution function, using the Jeffery distribution function, Eqn.~(2.103).  Uses \texttt{A2F} and \texttt{matchA}. }

        \item[A2Faccuracy.mlx]{explains how to control the accuracy of \texttt{A2F}, which finds the deformation gradient tensor $\F$ that will transform an initially isotropic orientation state to a given second-order tensor $\A$, using the deformation form of Jeffery's equation.  \texttt{F2A} does the reverse calculation.}    
        
\end{description}

\subsection*{Chapter 3.  Measuring Fiber Orientation and Length}

\begin{description}
    
    \item[PlanarSectionMeasurement.mlx]{uses \texttt{thetaphi2A} to compute orientation information for data from a planar section.  Follows Example~3.1.1.}
      
\end{description}

\subsection*{Chapter 4. Flow Orientation of Single Fibers}

\begin{description}

    \item[JefferyFiberMotion.mlx]{calculates the motion of a single fiber following Jeffery's equation using \texttt{pDot} and \texttt{ode45} (Section~4.2.3).   Jeffery orbits are illustrated, as in Fig.~4.14.}

    \item[JefferyDeformationForm.mlx]{illustrates the use of \texttt{changep}, which implements the deformation form of Jeffery's equation.  Also uses \texttt{randomfibers} to generate a set of $\pv$ vectors that are randomly distributed in all directions, and \texttt{p2A} to determine the initial and final orientation tensors.}
        
\end{description}


\subsection*{Chapter 5. Flow Orientation of Groups of Fibers} 

\begin{description}    

        \item[JefferyTensorEqn.mlx]{illustrates the numerical solution of the orientation tensor equation when every fiber follows Jeffery's equation.  This is the example from Section 5.1.2, and uses \texttt{AdotJeffQuad}.}    

        \item[SolvePlanarDistnFcn.mlx]{shows how to solve for the planar orientation distribution function $\psi_\phi (\phi, t)$ using \texttt{solvePsi2D}.  See Sections~5.2.1 and 5.3.1.  This script produces Fig.~5.5(a), and illustrates the control of `wiggles' in the finite difference solution using power-law upwinding and/or grid refinement.}    

        \item[Solve3DdistnFcn.mlx]{shows how to solve for the 3-D orientation distribution function $\psi (\pv, t)$  using \texttt{solvePsi3D}.  See Sections~5.2.2 and 5.3.2.  This script produces the tensor history in Fig.~5.7, and shows how to display distribution functions as in Fig. 5.6.  This script also shows how to calculate $\psi(\pv, t)$ for anisotropic rotary diffusion models using \texttt{solvePsiARD} (Section~ 5.6).  }    

        \item[OrientationTensorEqns.mlx]{uses \texttt{Adot2} together with \texttt{ode45} to solve tensor equations for flow-induced fiber orientation.  The script shows solutions to the Folgar-Tucker equation (Sections~5.3 and 5.4), anisotropic rotary diffusion models (Section~5.6), and slow kinetics models (Section~5.7).}    
 
        \item[FitCI.mlx]{illustrates the use of \texttt{fitCI} to find the interaction coefficient $C_I$ that produces a given steady-state value of $A_{11}$ in simple shear flow for the Folgar-Tucker model.  This script also demonstrates that the choice of closure approximation affects the value of $C_I$.}   
 
        \item[FitARDparams.mlx]{uses \texttt{fitARD} to obtain the ARD parameters that give desired steady-state values of $A_{11}$ and $A_{33}$ in simple shear flow.  Also demonstrates that the steady-state orientation is independent of the kinetic parameter $\kappa$ when an RSC or RPR model is used together with anisotropic rotary diffusion.}   
        
\end{description}

\subsection*{Chapter 6. Suspension Rheology and Flow-Orientation Coupling}

\begin{description}

    \item[FiberSuspensionStress.mlx]{demonstrates the use of \texttt{tauFiber} to find the orientation-dependent stress in a fiber suspension.  Shows how to do the calculations used to generate Fig.~6.4.}
    
\end{description}

\subsection*{Chapter 7. Fiber Length Degradation during Processing}

\begin{description}

    \item[FiberLengthModel.mlx]{explains the dimensionless version of the Phelps-Tucker fiber length model as implemented in \texttt{solveFLDstar}, and shows the calculations for the example in Fig. 7.4.}

\end{description}

\subsection*{Chapter 8. Mechanical Properties and Orientation}

\begin{description}

    \item[UnidirectionalProperties.mlx]{shows how to calculate the stiffness and thermal expansion of composites with unidirectional fiber orientation using mean-field theories.  The main functions for this are \texttt{mori} for the Mori-Tanaka model, \texttt{lielens} for the Lienlens/double-inclusion model, and \texttt{halpin} for the Halpin-Tsai model.  The functions \texttt{iso2C}, \texttt{C2eng}, and \texttt{inv4} are also used.}
    
    \item[OrientationAveragedProperties.mlx]{applies orientation averaging to find the properties of a composite with a distribution of fiber orientation.  Both stiffness and thermal expansion are considered, and Voigt averages are computed.  The main functions are \texttt{oravg4} and \texttt{oravg2}.  The script reproduces Fig.~8.5(a) showing elastic modulus vs.\ orientation, and creates a plot of thermal expansion vs.\ orientation similar to Fig.~8.8(a).}
        
    \item[LaminatedPlateProperties.mlx]{uses classical lamination theory, as implemented in \break \texttt{Clayer2laminate}, to find the tensile and flexural moduli of an injection molded composite whose orientation varies across the thickness.  Partial results from Table~8.4 are reproduced.  A summary of the underlying theory is given in \textbf{LaminationTheory.pdf} in the Documentation folder.}
\end{description}



\subsection*{Journal article: Planar Orientation Distributions}

\begin{description}
    
    \item[PlanarDistnExamples.mlx]{calculates and plots example distribution functions for planar orientation.  These include the Jeffery, ER and Bingham distributions, as well as several fourth-order maximum entropy distributions.}
    
    \item[CreateNonOrthoClosure.mlx]{shows how to use \texttt{fitNonOrtho2D} to generate data for a variety of flow cases, fit a non-orthotropic (F) closure to the data, and add that closure to \texttt{closeA4planar}.}
    
    \item[TestPlanarClosure.mlx]{compares transient results from a distribution function calculation to several different closure approximations.  Both the tensor components and the scalar error measure are plotted.}
    
    \item[TestPlanarReconstruction.mlx]{compares different reconstruction strategies for planar distributions (Jeffery; Bingham; ER; and ME4 with the natural, ER, and F closures and the exact fourth-order tensor) against distribution functions calculated using the Folgar-Tucker model.  Transient errors are plotted, as well as the calculated and reconstructed distributions at the end time.  While the book recommended the Jeffery distribution for planar reconstruction, the paper shows that the fourth-order maximum entropy distribution (ME4) with the best available $\Afour$ information is a better choice.}
    
        
\end{description}

\section{Functions Listed by Category}

\subsection{Operations on Orientation Tensors}

Starting in version~1.1, all functions operate on either 3-D or 2-D (planar) tensors, except as noted.  The functions determine whether to return 2-D or 3-D results based on the dimensions of the input matrices.  

\begin{description}

    \item[eigsort]{returns the eigenvalues and eigenvectors of a second-order tensor, sorted from largest eigenvalue to smallest.}

    \item[inv4]{finds the tensor inverse of a fourth-order tensor, in $6 \times 6$ matrix form for 3-D or $3 \times 3$ in 2-D.}

    \item[p2A]{converts a set of $\pv$ vectors, with or without weighting factors, to second-order and fourth-order orientation tensors.}

    \item[rotate4]{performs a coordinate transformation on a fourth-order tensor.  See Section~A.4.5.}

    \item[tens2vec]{converts a symmetric second-order tensor from $3 \times 3$ matrix form to $6 \times 1$ column vector (contracted) form.  See Section~2.3.5.1.  In 2-D the matrix is $2 \times 2$ and the vector is $3 \times 1$.}

    \item[tens2vec4]{converts a symmetric fourth-order tensor from $6 \times 6$ matrix form to $15 \times 1$ column vector form.  This is not discussed in the book, but is used by \texttt{matchA} in 3-D, and for 2-D distribution function recovery.  In 2-D the matrix is $3 \times 3$ and the vector is $5 \times 1$.}

    \item[thetaphi2A]{returns the second-order orientation tensor corresponding to a set of angles $(\theta, \phi)$ measured from planar section data (Section~3.1.1).  Either the Bay or Konicek weighting functions can be used.  3-D orientation only.}

    \item[transisoA]{returns the full second-order orientation tensor $\A$ and fourth-order orientation tensor $\Afour$ for a transversely isotropic orientation state with given values of $A_{11}$ and $\Afour_{1111}$.  3- D orientation only.}

    \item[vec2tens]{converts a symmetric second-order tensor from $6 \times 1$ column vector form to $3 \times 3$ matrix form.  See Section~2.3.5.1.  In 2-D the vector is $3 \times 1$ and the matrix is $2 \times 2$.}

    \item[vec2tens4]{converts a symmetric fourth-order tensor from $15 \times 1$ column vector form to $6 \times 6$ matrix form.  In 2-D the vector is $5 \times 1$ and the matrix is $3 \times 3$  This is not discussed in the book, but is the inverse of \texttt{tens2vec4}.  It can be used to convert fourth-order tensor results from \texttt{solvePsi3D} and \texttt{solvePsiARD} to matrix form.}
    
\end{description}

\subsection{Flow-Induced Orientation Models}

\begin{description}

    \item[Adot2]{gives the time derivative $\dot{\A}$ as a function of $\A$, $\mathbf{L}$ and orientation model parameters, for a wide range of orientation models.  This is the principal tool used to predict 3-D flow-induced orientation.  The 2 in the name denotes a second-generation version of this function.}
    
    \item[AdotPlanar]{gives the time derivative $\dot{\A}$ as a function of $\A$, $\mathbf{L}$ and orientation model parameters for planar orientation.}
    
    \item[AdotJeffQuad]{gives the time derivative $\dot{\A}$ for the 3-D Jeffery model using the quadratic closure.  This is a simplified version of \texttt{Adot2}, used in Section~5.1.2 and Fig.~5.1.}
    
    \item[Asteady]{finds the steady-state 3-D orientation tensor $\A$ for a given velocity gradient $\mathbf{L}$, for any orientation model in \texttt{Adot2}.}
    
    \item[AsteadyPlanar]{finds the steady-state planar orientation tensor $\A$ for a given velocity gradient $\mathbf{L}$, for any orientation model in \texttt{AdotPlanar}.}
    
    \item[changep]{finds a set of current orientation vectors $\pv$ using the deformation form of Jeffery's equation, for a given set of initial vectors $\pv'$ and deformation gradient tensor $\F$.}
    
    \item[closeA4]{uses any of several 3-D closure approximations to find the fourth-order orientation tensor $\Afour$ corresponding to a second-order tensor $\A$.  See Section~ 5.4.}  
    
    \item[closeA4planar]{uses any of several closure approximations to find the planar fourth-order orientation tensor $\Afour$ corresponding to a planar second-order tensor $\A$.  This includes the non-orthotropic F closures.}  
    
    \item[fitARD]{finds the 3-D anisotropic rotary diffusion (ARD) model parameters that achieve given steady-state values of $A_{11}$ and $A_{33}$ in 1--3 simple shear flow.}  
    
    \item[fitCI]{finds the interaction coefficient $C_I$ for the 3-D Folgar-Tucker model that achieves a given steady-state value of $A_{11}$ in simple shear flow.}
    
    \item[fitNonOrtho2D]{determines the polynomial coefficients for a non-orthotropic, planar closure (F closure) for a given interaction coefficient and set of flow cases.  The utility function \texttt{printCoeff} is useful for formatting the results for pasting into \texttt{closeA4planar}.}
    
    \item[fitOrtho2D]{determines the polynomial coefficients for an orthotropic, planar closure (D closure) for a given interaction coefficient.  The utility function \texttt{printCoeff} is useful for formatting the results for pasting into \texttt{closeA4planar}.}
    
    \item[pDot]{gives the time derivative $\dot{ \pv}$ of a 3-D orientation vector $\pv$ using Jeffery's equation.  See Section~4.2.3.}
    
    \item[solvePsi2D]{solves for the orientation distribution function $\psi_\phi ( \phi, t)$ for the 2-D version of the Folgar-Tucker model.  See Sections~5.2.1 and 5.3.21}
    
    \item[solvePsi3D]{solves for the orientation distribution function $\psi( \theta, \phi, t)$ for the 3-D version of the Folgar-Tucker model.  See Sections~5.2.2 and 5.3.2.}
    
    \item[solvePsiARD]{solves for the orientation distribution function $\psi( \theta, \phi, t)$ for 3-D anisotropic rotary diffusion models.  See Section~5.6.}
    
    \item[tauFiber]{finds the extra-stress tensor $\btau$ for a fiber suspension, for a given 3-D orientation tensor $\A$ and rate of deformation $\D$, Eqn.~(6.29).}
    
\end{description}

\subsection{Fiber Length Prediction}

\begin{description}

    \item[fldRstar]{returns a matrix $[R^*]$ used in the non-dimensional version of the Phelps-Tucker fiber length model.  This function is not used directly, but is required by \texttt{solveFLDstar}.}
    
    \item[solveFLDstar]{solves a non-dimensional version of the Phelps-Tucker fiber length model.}
    
\end{description}

\subsection{Mechanical Property Prediction}

\begin{description}

    \item[C2eng]{finds the engineering constants for a given stiffness tensor $\Cfour$.}

    \item[Clayer2laminate]{finds the laminate stiffness matrices $[A]$, $[B]$, and $[D]$ for a laminate, given the stiffness tensor $\Cfour$ for each layer and the layer thicknesses.  This is used to compute the tensile and bending properties of a composite where the orientation varies across the thickness; see Section~8.4.5.}
    
    \item[diluteEshelby]{finds the stiffness tensor $\Cfour$ for a dilute composite with unidirectional alignment using Eshelby's equivalent inclusion.  This model is primarily of theoretical interest, and is used in Fig.~8.3.}    
    
    \item[eng2C]{converts the engineering constants for an orthotropic material into a stiffness tensor $\Cfour$.  See also \texttt{iso2C}.}    
    
    \item[eshtens]{returns the Eshelby tensor $\Efour$ for a spheroidal particle in an isotropic matrix.  The particle can be prolate (fiber-like), spherical, or oblate (disk-like).}
    
    \item[halpin]{finds the engineering constants, stiffness tensor $\Cfour$, and thermal stress tensor $\bbeta$ for a discontinuous fiber composite with unidirectional fibers using the Halpin-Tsai equations.}    
    
    \item[iso2C]{finds the stiffness tensor $\Cfour$ for an isotropic material with Young's modulus $E$ and Poisson ratio $\nu$.}    
    
    \item[lielens]{returns the stiffness tensor $\Cfour$ and thermal stress tensor $\bbeta$ for a unidirectional composite using the Lielens/double-inclusion model.}    
    
    \item[mori]{returns the stiffness tensor $\Cfour$ and thermal stress tensor $\bbeta$ for a unidirectional composite using the Mori-Tanaka model.}   
    
    \item[oravg2]{computes the 3-D orientation average of a transversely isotropic second-order tensor, Eqn.~(8.81).}   
    
    \item[oravg4]{computes the 3-D orientation average of a transversely isotropic fourth-order tensor, Eqn.~(8.77).}   
    
\end{description}

\subsection{Reconstruction of Orientation Distribution Functions}

\begin{description}

    \item[A2F]{finds the 3-D deformation gradient tensor $\F$ for any orientation tensor $\A$ using the deformation form of Jeffery's equation.}
    
    \item[F2A]{returns the 3-D orientation tensor $\A$ for a given deformation gradient tensor $\F$ using Jeffery's model.}
    
    \item[fitBingham2D]{finds the Bingham distribution that matches a given planar second-order orientation tensor $\A$.}
    
    \item[fitERdistn2D]{finds the ellipse radius (ER) distribution that matches a given planar second-order orientation tensor $\A$.}
    
    \item[Jefferydistn2D]{finds the Jeffery distribution that matches a given planar second-order orientation tensor $\A$.}
    
    \item[fitMaxEntropy2D]{finds the fourth-order maximum entropy distribution (ME4) that matches a given planar fourth-order orientation tensor $\Afour$.}
    
    \item[matchA,]{given a set of 3-D orientation vectors $\pv^i$ and weights $f^i$, adjusts the weights to exactly match a given second-order or fourth-order orientation tensor, while minimizing the mean square difference between the original and adjusted weights.  WARNING: This function can return negative values for some weights, usually when the orientation state is highly aligned.  See \texttt{RecontructDiscreteDistributionFcn.mlx} for examples.}    
    
\end{description}

\subsection{Graphics and Utility Functions}

\begin{description}
    
    \item[drawPsi]{draws a sphere colored by the 3-D Jeffery orientation distribution function for a given deformation gradient tensor $\F$.}

    \item[fill3elt]{draws a triangular mesh in three dimensions, colored according to element values.}

    \item[fill3mesh]{draws a triangular mesh in three dimensions, colored according to nodal values.}

    \item[meshcon]{builds the edge connectivity information for a triangular mesh, as needed by \texttt{refinemesh}.}

    \item[p2sph]{converts a set of 3-D $\pv$ vectors to the angles $(\theta, \phi)$ in a spherical coordinate system.}

    \item[plot3mesh]{draws the elements of a triangular mesh in three dimensions.}

    \item[plot3nodes]{draws the nodes of a mesh in three dimensions.}

    \item[polyfit2]{finds a polynomial function of two variables $\hat{z}(x, y)$ that best fits a set of data $z_i(x_i, y_i)$.  Equality constraints can be enforced if desired.}

    \item[polyfit2]{evaluates a polynomial function of two variables $\hat{z}(x, y)$ using coefficients in the form returned by \texttt{polyfit2}.}

    \item[printCoeff]{formats a vector of polynomial coefficients and prints them to the console.  From there, they can then be copied and pasted into a \textsc{Matlab} code such as \texttt{closeA4planar}.}
    
    \item[randomfibers]{generates a set of fiber orientation vectors $\pv$, randomly oriented in three dimensions.}

    \item[refinemesh]{refines a triangular mesh by dividing each initial element into $n^2$ smaller ones.}

    \item[sph2p]{converts a set of angle pairs $(\theta, \phi)$ in a spherical coordinate system into unit vectors $\pv$.}

    \item[spheremesh]{generates different types of triangular meshes on a unit sphere, or on half of a sphere.}

    \item[sphTriArea]{finds the area of each triangle in a mesh on the unit sphere.  Each element is treated as a spherical triangle, rather than a planar triangle.}

    \item[surfPsi3D]{colors the surface of a unit sphere according to an orientation distribution function $\psi (\pv)$.  This function is designed to display distribution functions calculated by \texttt{solvePsi3D} and \texttt{solvePsiARD}.}

    \item[weightFrac2volFrac]{converts fiber weight fraction to fiber volume fraction for a two-phase composite.}
    
    
\end{description}





\end{document}